
\documentclass[conference]{IEEEtran}

%\usepackage{ifpdf}
% \ifpdf
%   % pdf code
% \else
%   % dvi code
% \fi
%\usepackage{cite}
\ifCLASSINFOpdf
   \usepackage[pdftex]{graphicx}
   \graphicspath{{../pdf/}{../jpeg/}}
   \DeclareGraphicsExtensions{.pdf,.jpeg,.png}
\else
   \usepackage[dvips]{graphicx}
   \graphicspath{{../eps/}}
   \DeclareGraphicsExtensions{.eps}
\fi
\usepackage{amsmath}
\usepackage{amsfonts}
\usepackage{amssymb}
\interdisplaylinepenalty=2500
%\usepackage{algorithmic}
%\usepackage{array}
%\ifCLASSOPTIONcompsoc
%  \usepackage[caption=false,font=normalsize,labelfont=sf,textfont=sf]{subfig}
%\else
%  \usepackage[caption=false,font=footnotesize]{subfig}
%\fi
%\usepackage{fixltx2e}
%\usepackage{stfloats}
%\fnbelowfloat
% \usepackage{dblfloatfix}
%\usepackage{url}
%\hyphenation{op-tical net-works semi-conduc-tor}
\usepackage[]{algorithm2e}
\usepackage[]{color}

\begin{document}

\title{Platooning II - This is a temporary title}

\maketitle

\suppressfloats

\begin{abstract}
    [ref to platooning paper] proposed a formulation for platooning
    UAVs to make reachability analysis on UAVs in traffic infrastructures
    tractable. In [ref to platooning paper], the theory and scenario studies, through
    virtual simulation, showcase the potential of the formulation for
    ensuring liveness and safety guarantees on vehicles in transit.
    To show the efficacy of platooning on real systems we
    implemented the formulation on real quadrotors. Due to the curse of
    dimensionality, HJ reachability-based control is rarely implemented for
    quadrotors. In this paper we present a novel hardware implementation
    architecture for platooning with both real UAV's and real information
    patterns. We demonstrate the efficacy of our implementation by showcasing
    the results of running the system through three scenario studies.
\end{abstract}

% \ifCLASSOPTIONpeerreview
% \begin{center} \bfseries EDICS Category: 3-BBND \end{center}
% \fi
%
\IEEEpeerreviewmaketitle

%%%%%%%%%%%%%%%%%%%%%%%%%%%%%%%%%%%%%%%%%%%%%%%%%%%%%%%%%%%%%%%%%%%%%%%%%%

\input{Intro}

%%%%%%%%%%%%%%%%%%%%%%%%%%%%%%%%%%%%%%%%%%%%%%%%%%%%%%%%%%%%%%%%%%%%%%%%%%

\input{Background}

%%%%%%%%%%%%%%%%%%%%%%%%%%%%%%%%%%%%%%%%%%%%%%%%%%%%%%%%%%%%%%%%%%%%%%%%%%
\section{Implementation}

    \input{Hardware}

    \input{Real_reach}

    \input{Controller}

%%%%%%%%%%%%%%%%%%%%%%%%%%%%%%%%%%%%%%%%%%%%%%%%%%%%%%%%%%%%%%%%%%%%%%%%%%

\input{Case_studies}

%%%%%%%%%%%%%%%%%%%%%%%%%%%%%%%%%%%%%%%%%%%%%%%%%%%%%%%%%%%%%%%%%%%%%%%%%%

\input{Conclusion}

%%%%%%%%%%%%%%%%%%%%%%%%%%%%%%%%%%%%%%%%%%%%%%%%%%%%%%%%%%%%%%%%%%%%%%%%%%

%\section*{Acknowledgment}

%%%%%%%%%%%%%%%%%%%%%%%%%%%%%%%%%%%%%%%%%%%%%%%%%%%%%%%%%%%%%%%%%%%%%%%%%

\begin{thebibliography}{1}

\bibitem{platooningI}
    ref to platooning I
\bibitem{cfnnet}
    ref to cfnnet
\bibitem{cfrepo}
    ref to wolfgang's repo
\bibitem{ROS}
    ref to ROS

\end{thebibliography}

%%%%%%%%%%%%%%%%%%%%%%%%%%%%%%%%%%%%%%%%%%%%%%%%%%%%%%%%%%%%%%%%%%%%%%%%%

\end{document}


