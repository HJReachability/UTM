\documentclass[submit]{aiaa-pretty}

\usepackage{todonotes} \setlength{\marginparwidth}{2.3cm} 

\newcommand{\MCnote}{\todo[size=\footnotesize,author=MC,color=cyan]}


\newcommand{\pos}{p} % position
\newcommand{\Pos}{\bar{p}} % a particular position (or desired position}
\newcommand{\vel}{v} % velocity
\newcommand{\Vel}{\bar{v}} % a particular position (or desired position}
\newcommand{\hw}{\mathbb{H}} % a single highway
\newcommand{\hws}{\mathbb{S}} % a sequence of highways
\newcommand{\hwd}{\hat{d}}
\newcommand{\cost}[1]{c_\text{#1}}
\newcommand{\wpt}{\mathcal{W}}
\newcommand{\cmap}{c}
\newcommand{\ccost}{C}
\newcommand{\ppath}{\mathbb{P}}
\newcommand{\ocost}{V}

\newcommand{\sepdist}{d_\text{sep}} % separation distance

\newcommand{\td}{t_\text{faulty}} % time to descend
\newcommand{\veh}[1]{Q_{#1}}
\newcommand{\vehSCS}[1]{\mathcal{Q}_{#1}} % vehicle safety check set

\author{Mo Chen, Qie Hu, Jaime Fisac, Kene Akametalu, Casey Mackin, Claire Tomlin}
\title{Safety and Liveness of Unmanned Aerial Vehicle Platoons on Air Highways}

\abstract{Recently, there has been immense interest in using unmanned aerial vehicles (UAVs) for civilian operations. As a result, unmanned aerial systems traffic management is needed to ensure the safety and liveness of thousands of UAVs flying simultaneously. Currently, the analysis of large multi-agent systems cannot tractably provide safety and liveness guarantees if the agents' set of maneuvers is unrestricted. In this paper, platoons of UAVs flying on air highways is proposed to impose an airspace structure that allows for tractable analysis. For the air highway placement problem, the fast marching method is used to produce a sequence of air highways that minimizes the cost of flying from an origin to any destination. The placement of air highways can be updated in real-time to accommodate sudden airspace changes. Within platoons that travel on air highways, each vehicle is modeled as a hybrid system. Using Hamilton-Jacobi reachability, safety and liveness are guaranteed for all mode transitions. For a single altitude range, the proposed approach guarantees safety for one safety breach per vehicle; in the unlikely event of multiple safety breaches, safety can be guaranteed over multiple altitude ranges. The satisfaction of safety and liveness requirements is demonstrated through simulations of three scenarios.}

\begin{document}
\maketitle
\section*{Nomenclature}
\noindent\begin{tabular}{@{}lcl@{}}
$\cmap$ &=& Cost map \\
$\ppath$ &=& A path between two points \\
$\ccost$ &=& Cumulative cost of a path \\
$\ocost$ &=& Value function of partial differential equations \\
$\hw$ &=& Air highway \\
$\hwd$ &=& Direction of travel of air highway \\
$\hws$ &=& A sequence of air highways \\
$\wpt$ &=& Waypoint \\
$x$ &=& System state (of a vehicle) \\
$\pos=(\pos_x, \pos_y)$ &=& Horizontal position \\
$\vel=(\vel_x, \vel_y)$ &=& Horizontal velocity \\
$\Pos$ &=& Target position \\
$\Vel$ &=& Target velocity \\
$\sepdist$ &=& Separation distance of vehicles within a platoon \\
$\td$ &=& Time limit for descent during potential conflict \\
$\veh{i}$ &=& $i$th vehicle \\
$\vehSCS{i}$ &=& Set of vehicles for vehicle $\veh{i}$ to consider for safety 
\end{tabular} \\

% !TEX root = main.tex
\section{Introduction}
Unmanned aerial vehicles (UAVs) have in the past been mainly used for military operations \cite{Tice91}; however, recently there has been an immense surge of interest in using UAVs for civil applications. Through projects such as Amazon Prime Air \cite{PrimeAir} and Google Project Wing \cite{ProjectWing}, companies are looking to send UAVs into the airspace to not only deliver commercial packages, but also for important tasks such as aerial surveillance, emergency supply delivery, videography, and search and rescue \cite{Kopardekar16}. In the future, the applications of UAVs are only limited by human imagination. 

As a rough estimate, suppose in a city of 2 million people, each person requests a drone delivery every 2 months on average and each delivery requires a 30-minute trip for a UAV. This would equate to thousands of UAVs simultaneously in the air just from package delivery services. Applications of UAVs extend beyond package delivery; they can also be used, for example, to provide supplies or to respond to disasters in areas that are difficult to reach but require prompt response \cite{Debusk10,Tornado16}. As a result, government agencies such as the Federal Aviation Administration (FAA) and National Aeronautics and Space Administration (NASA) are also investigating unmanned aerial systems (UAS) traffic management (UTM) in order to prevent collisions among potentially numerous UAVs \cite{Kopardekar16, FAA13, NASA16}. 

Optimal control and game theory are powerful tools for providing liveness and safety guarantees to controlled dynamical systems under bounded disturbances, and various formulations \cite{Bokanowski10,Mitchell05,Barron89} have been successfully used to analyze problems involving small numbers of vehicles \cite{Fisac15,Chen14,Ding08}. These formulations are based on Hamilton-Jacobi (HJ) reachability, which computes the reachable set, defined as the set of states from which a system is guaranteed to have a control strategy to reach a target set of states. Reachability is a powerful tool because reachable sets can be used for synthesizing both controllers that steer the system towards a set of goal states (liveness controllers), and controllers that steer the system away from a set of unsafe states (safety controllers). Furthermore, the HJ formulations are flexible in terms of system dynamics, enabling the analysis of non-linear systems. The power and success of HJ reachability analysis in previous applications cannot be denied, especially since numerical tools are readily available to solve the associated HJ partial differential equation (PDE) \cite{LSToolbox,Osher02,Sethian96}. However, the computation is done on a grid, making the problem complexity scale exponentially with the number of states, and therefore with the number of vehicles. Consequently, HJ reachability computations are intractable for large numbers of vehicles. 

In order to accommodate potentially thousands of vehicles simultaneously flying in the air, additional structure is needed to allow for tractable analysis and intuitive monitoring by human beings. An air highway system on which platoons of vehicles travel accomplishes both goals. However, many details of such a concept need to be addressed. Due to the flexibility of placing air highways compared to building ground highways in terms of highway location, even the problem of air highway placement can be a daunting task. To address this, in the first part of this paper, we propose a flexible and computationally efficient method based on \cite{Sethian96} to perform optimal air highway placement given an arbitrary cost map that captures the desirability of having UAVs fly over any geographical location. We demonstrate our method using the San Francisco Bay Area as an example. Once air highways are in place, platoons of UAVs can then fly in fixed formations along the highway to get from origin to destination.

A considerable body of work has been done on the platooning of ground vehicles \cite{Kavathekar11}. For example, \cite{McMahon90} investigated the feasibility of vehicle platooning in terms of tracking errors in the presence of disturbances, taking into account complex nonlinear dynamics of each vehicle. \cite{Hedrick92} explored several control techniques for performing various platoon maneuvers such as lane changes, merge procedures, and split procedures. In \cite{Lygeros98}, the authors modeled vehicles in platoons as hybrid systems, synthesized safety controllers, and analyzed throughput. Reachability analysis was used in \cite{Alam11} to analyze a platoon of two trucks in order to minimize drag by minimizing the following distance while maintaining collision avoidance safety guarantees. Finally, \cite{Sabau16} provided a method for guaranteeing string stability and eliminating accordion effects for a heterogeneous platoon of vehicles with linear time-invariant dynamics.

Previous analyses of a large number of vehicles typically do not provide liveness and safety guarantees to the extent that HJ reachability does; however, HJ reachability typically cannot be used to tractably analyze a large number of vehicles. In the second part of this paper, we propose organizing UAVs into platoons, which provide a structure that allows pairwise safety guarantees from HJ reachability to better translate to safety guarantees for the whole platoon. With respect to platooning, we first propose a hybrid systems model of UAVs in platoons to establish the modes of operation needed for our platooning concept. Then, we show how reachability-based controllers can be synthesized to enable UAVs to successfully perform mode switching, as well as prevent dangerous configurations such as collisions. Finally, we show several simulations to illustrate the behavior of UAVs in various scenarios.
% !TEX root = main.tex
\section{Air Highways}
\subsection{Problem Formulation}
% !TEX root = main.tex
\subsection{The Eikonal Equation -- Cost-Minimizing Path}
Let $s_0, s_1\in \mathbb{R}$, and let $\ppath: [s_0, s_1] \rightarrow \mathbb{R}^2$ be a path starting from an origin point $\pos^o = \ppath(s_0)$ and ending at a destination point $\pos^d = \ppath(s_1)$. Note that the sequence $\hws_N$ in \eqref{eq:ahpp} is a piece-wise affine example of a path $\ppath(s), s\in[s_0, s_1]$; however, a path $\ppath$ that is not piece-wise affine cannot be written as a sequence of highways $\hws_N$.

More concretely, suppose a UAV flies from an origin point $p^o$ to a destination point $p^d$ along some path $\ppath(s)$ parametrized by $s$. Then, $\ppath(s_0) = p^o$ would denote the origin, and $\ppath(s_1) = p^d$ would denote the destination. All intermediate $s$ values denote the intermediate positions of the path, i.e. $\ppath(s) = \pos(s) = (\pos_x(s), \pos_y(s))$.

Consider the cost map $\cmap(\pos_x, \pos_y)$ which captures the cost incurred for UAVs flying over the position $\pos = (\pos_x, \pos_y)$. Along the entire path $\ppath(s)$, the cumulative cost $\ccost(\ppath)$ is incurred. Define $\ccost$ as follows:

\begin{equation}
\ccost(\ppath) = \int_{s_0}^{s_1} \cmap(\ppath(s)) ds
\end{equation}

For an origin-destination pair, we would like to find the path such that the above cost is minimized. More generally, given an origin point $p^o$, we would like to compute the function $\ocost$ representing the optimal cumulative cost for any destination point $\pos^d$:

\begin{equation}
\label{eq:rahpp} % relaxed air highway placement problem
\begin{aligned}
\ocost(\pos^d) &= \min_{\ppath(\cdot)} \ccost(\ppath) \\
&= \min_{\ppath(\cdot)} \int_{s_0}^{s_1} \cmap(\ppath(s)) ds
\end{aligned}
\end{equation}

It is well known that the solution to the Eikonal equation \eqref{eq:eikonal} precisely computes the function $\ocost(\pos^d)$ given the cost map $\cmap$ \cite{Sethian96,Alton06}. Note that a single function characterizes the minimum cost from an origin $\pos^o$ to \textit{any} destination $\pos^d$. Once $\ocost$ is found, the optimal path $\ppath$ between $\pos^o$ and $\pos^d$ can be obtained via gradient descent.

\begin{equation}
\label{eq:eikonal}
\begin{aligned}
\cmap(\pos)|\nabla \ocost(\pos)| &= 1 \\
\ocost(\pos^o) &= 0
\end{aligned}
\end{equation}

The Eikonal equation \eqref{eq:eikonal} can be efficiently computed numerically using the fast marching method \cite{Sethian96}.

Note that \eqref{eq:rahpp} can be viewed as a relaxation of the air highway placement problem defined in \eqref{eq:ahpp}. Unlike \eqref{eq:ahpp}, the relaxation \eqref{eq:rahpp} can be quickly solved using currently available numerical tools. Thus, we first solve the approximate air highway placement problem \eqref{eq:rahpp} by solving \eqref{eq:eikonal}, and then post-process the solution to \eqref{eq:rahpp} to obtain an approximation to \eqref{eq:ahpp}.

Given a single origin point $\pos^o$, the optimal cumulative cost function $\ocost(\pos^d)$ can be computed. Suppose $M$ different destination points $\pos^d_i,i=1,\ldots,M$ are chosen. Then, $M$ different optimal paths $\ppath_i,i=1,\ldots,M$ are obtained from $\ocost$.
% !TEX root = main.tex
\subsection{From Paths to Waypoints}
Each of the cost-minimizing paths $\ppath_i$ computed from the solution to the Eikonal equation consists of a continuous set of points. Each path $\ppath_i$ is an approximation to the sequence of highways $\hws_{N_i}^i = \{\hw^i_j\}_{i=1,j=1}^{i=M,j=N_i}$ defined in \eqref{eq:ahpp}, but now indexes by the corresponding path. 

For each path $\ppath_i$, we would like to sparsify the points on the path to obtain a collection of waypoints, $\wpt_{i,j}, j = 1,\ldots, N_i+1$, which are the end points of the highways:

\begin{equation}
\begin{aligned}
\hw^i_j(0) &= \wpt_{i,j}, \\
\hw^i_j(1) &= \wpt_{i,j+1}, \\
j &= 1,\ldots,N_i
\end{aligned}
\end{equation}

 we sparsify this set of points to obtain a collection of waypoints, $\wpt$, which are also endpoints of air highways, we first add the destination point to the collection and note the path's heading. Next, we add to the collection of waypoints the first point on the path at which the heading changes by some threshold $\theta_C$. This process is repeated along the entire path. Finally, we add the origin points to the collection. We create a collection of these points for all the cost-minimizing paths.

If there is a large change in heading within a small section of the cost-minimizing path, then the collections $\wpt_{ij}$ may contain many points which are close together. In addition, there may be multiple paths that are very close to each other (in fact, this behavior is desirable), also cluttering the airspace with too many waypoints. We propose to sparsify the waypoints by clustering the points in $\wpt_{ij}$. Each cluster contains points that are within a certain distance to the closest point in the cluster, and all points in each cluster are replaced with a single point located at the centroid of the cluster.
% !TEX root = main.tex
\subsection{Results}
% !TEX root = main.tex
\section{Unmanned Aerial Vehicle Platooning \label{sec:platooning}}
\subsection{Problem Formulation}
\subsubsection{Vehicle Dynamics}
Consider a UAV whose dynamics are given by
\begin{equation}
\dot{x} = f(x,u)
\end{equation}

\noindent where $x$ represents the state, and $u$ represents the control action. In this paper, we will assume that each vehicle has the following simple model of a quadrotor:

\begin{equation} \label{eq:dyn}
\begin{aligned}
\dot{\pos}_x &= \vel_x \\
\dot{\pos}_y &= \vel_y  \\
\dot{\vel}_x &= u_x \\
\dot{\vel}_y &= u_y \\
|u_x|,|u_y| &\le u_\text{max}
\end{aligned}
\end{equation}

\noindent where the state $x=(\pos_x, \vel_x, \pos_y, \vel_y)\in\mathbb{R}^4$ represents the quadrotor's position in the $x$-direction, its velocity in the $x$-direction, and its position and velocity in the $y$-direction, respectively. The control input $u = (u_x, u_y)\in\mathbb{R}^2$ consists of the acceleration in the $x$- and $y$- directions. For convenience, we will denote the position and velocity $\pos=(\pos_x, \pos_y),\vel=(\vel_x,\vel_y)$, respectively. We will consider a group of $N$ quadrotors $Q_i, i=1\ldots,N$.

In general, the problem of collision avoidance among $N$ vehicles cannot be tractably solved using traditional dynamic programming approaches because the computation complexity of these approaches scales exponentially with the number of vehicles. Thus, in our present work, we will consider the situation where UAVs travel on air highways in platoons, defined in the following sections. The structure imposed by air highways and the platoon enables us to analyze the liveness and safety of the vehicles in a tractable manner.

\subsubsection{Vehicles as Hybrid Systems}
We model each vehicle as a hybrid system \cite{Lygeros98,Lygeros12} consisting of the modes "Free", "Leader", "Follower", and "Faulty". Within each mode, a set of maneuvers is available to allow the vehicle to change modes if desired. The modes and maneuvers are as follows:

\begin{itemize}
\item Free: 

A Free vehicle is not in a platoon or on a highway, and its possible maneuvers or mode transitions are
\begin{itemize}
\item remain a Free vehicle by staying away from highways,
\item become a Leader by entering a highway to create a new platoon, and
\item become a Follower by joining a platoon that is currently on a highway.
\end{itemize} 

\item Leader: 
A Leader vehicle is the vehicle at the front of a platoon (which could consist of only the vehicle itself). The available maneuvers and mode transitions are

\begin{itemize}
\item remain a Leader by traveling along the highway at a pre-specified speed,
\item become a Follower by merging the current platoon with a platoon in front, and
\item become a Free vehicle by leaving the highway.
\end{itemize}

\item Follower: 

A Follower vehicle is a vehicle that is following a platoon leader. The Available maneuvers and mode transitions are 

\begin{itemize}
\item remain a Follower by staying a distance of $b$ behind the vehicle in front in the current platoon,
\item become a Leader by splitting from the current platoon, and
\item become a Free vehicle by leaving the highway.
\end{itemize}

\item Faulty: 

If a vehicle from any of the other modes malfunction, it transitions into the Faulty mode and descends after a duration of $t_\text{internal}$.
\end{itemize}

The available maneuvers and associated mode transitions are summarized in Figure \ref{fig:vehicleModes}.

\begin{figure}
	\centering
	\includegraphics[width=0.5\textwidth]{"fig/vehicleModes"}
	\caption{Hybrid modes for vehicles in platoons.}
	\label{fig:vehicleModes}
\end{figure}

Suppose that there are $N$ vehicles in total. We consider a platoon of vehicles to be a group of $M\le N$ vehicles, denoted $Q_{P_1}, \ldots, Q_{P_M}, \{P_j\}_{j=1}^M \subseteq \{i\}_{i=1}^N$, in a single-file formation. We will assume that the vehicles in a platoon travel along an air highway. The vehicles maintain a separation distance of $b$ with its neighbors inside the platoon. In order to allow for close proximity of the vehicles and the ability to resolve multiple simultaneous safety breaches, we assume that in the event of a malfunction, a vehicle will be able to exit the altitude range of the highway within a duration of $t_\text{internal}=1.5$. Such a requirement may be implemented practically as an emergency landing procedure to which the vehicles revert when a malfunction is detected.

\subsubsection{Objectives}
Given the above modeling assumptions, our goal is to provide control strategies to guarantee the success and safety of all the mode transitions. The theoretical tool used to provide the liveness and safety guarantees is reachability. The reachable sets we compute will allow each vehicle to perform actions such as 

OK SHOULD EVASION BE A MODE?

\begin{itemize}

\item reacting to a malfunctioning vehicle in the platoon,
\item reacting to an intruder vehicle,
\item following the highway, a curve defined in space at constant altitude, at a specified speed, and
\item maintaining a constant relative position and velocity with the leader of a platoon.
\end{itemize}


\begin{enumerate}
\item How do vehicles form platoons?
\item How can the safety of the vehicles be ensured during normal operation and when there is a malfunctioning vehicle within the platoon?
\item How can the platoon respond to intruders such as unresponsive UAVs, birds, or other aerial objects?
\end{enumerate}

The answers to these questions can be broken down into the maneuvers listed in Section \ref{subsec:platoon_def}. In general, the control strategies of each vehicle have a liveness component, which specifies a set of states towards which the vehicle aims to reach, and a safety component, which specifies a set of states that it must avoid. Together, the liveness and safety controllers guarantee the success and safety of a vehicle in the airspace making any desired mode transition. In this paper, these guarantees are provided using reachability analysis.

% !TEX root = main.tex
\subsection{Hamilton-Jacobi Reachability}

% !TEX root = main.tex
\subsection{Reachability-Based Controllers \label{sec:reach_ctrl}}
Reachability analysis is useful for constructing controllers in a large variety of situations. In order to construct different controllers, an appropriate target set needs to be defined depending on the goal of the controller. If one defines the target set to be a set of desired states, the reachable set, once computed, would represent the states from which a system needs to first arrive at in order to reach the desired states. On the other hand, if the target set represents a set of undesirable states, then the reachable set would indicate the region of the state space that the system needs to avoid. In addition, the system dynamics with which the reachable set is computed provides additional flexibility when using reachability to construct controllers.

Using a number of different target sets and dynamics, we now propose different reachability-based controllers used for vehicle mode transitions in our platooning concept.

\subsubsection{Getting to a Target State \label{sec:abs_target_ctrl}}
The controller used by a vehicle to reach a target state is important in a couple of situations in the platooning context. First, a vehicle in the ``Free'' mode can use the controller to merge onto a highway, forming a platoon and changing modes to a ``Leader'' vehicle. Second, a vehicle in either the ``Leader'' mode or the ``Follower'' mode can use this controller to change to a different highway, changing modes to a ``Leader'' vehicle. 

In both of the above cases, we use the dynamics of a single vehicle specified in \eqref{eq:dyn}. The target state would be a position $(\Pos_x, \Pos_y)$ representing the desired merging point on the highway, along with a velocity $(\Vel_x, \Vel_y)$ that corresponds to the speed and direction of travel specified by the highway. For the reachability computation, we define the target set to be a small range of states around the target state $x_H = (\Pos_x, \Pos_y, \Vel_x, \Vel_y)$:

\begin{equation}
\begin{aligned}
\mathcal{L}_H = \{x: |\pos_x-\Pos_x|\le r_{\pos_x}, |v_x-\Vel_x|\le r_{\vel_x}, \\
|\pos_y - \Pos_y| \le r_{\pos_y}, |v_y - \Vel_y|\le r_{\vel_y} \}.
\end{aligned}
\end{equation}

Here, we represent the target set $\mathcal{L}_H$ as the zero sublevel set of the function $l_H(x)$, which specifies the terminal condition of the HJB PDE that we need to solve. Once the HJB PDE is solved, we obtain the reachable set $\mathcal V_H(t)$ from the subzero level set of the solution $V_H(t,x)$. More concretely, $\mathcal{V}_H(T) = \{x: V_H(-T,x)\le 0\}$ is the set of states from which the system can be driven to the target $\mathcal{L}_H$ within a duration of $T$. 

Depending on the time horizon $T$, the size of the reachable set $\mathcal V_H(T)$ varies. In general, a vehicle may not initially be inside the reachable set $\mathcal V_H(T)$, yet it needs to be in order to get to its desired target state. Determining a control strategy to reach $\mathcal V_H(T)$ is itself a reachability problem (with $\mathcal V_H(T)$ as the target set), and it would seem like this reachability problem needs to be solved in order for us to use the results from our first reachability problem. However, practically, one could choose $T$ to be large enough to cover a sufficiently large area to include any practically conceivable initial state. From our simulations, a suitable algorithm for getting to a desired target state is as follows:

\begin{enumerate}
\item Move towards $\bar{x}_H$ in a straight line, with some velocity, until $V_H(-T,x)\le 0$. In practice, this step consistently drives the system into the reachable set.
\item Apply the optimal control extracted from $V_H(-T,x)$ according to \eqref{eq:HJB_ctrl_syn} until $\mathcal{L}_H$ is reached.
\end{enumerate}

\subsubsection{Getting to a State Relative to Another Vehicle \label{sec:rel_target_ctrl}}
In the platooning context, being able to go to a state relative to another moving vehicle is important for the purpose of forming and joining platoons. For example, a ``Free'' vehicle may join an existing platoon that is on a highway and change modes to become a ``Follower''. Also, a ``Leader'' or ``Follower'' may join another platoon and afterwards go into the ``Follower'' mode.

To construct a controller for getting to a state relative to another vehicle, we use the relative dynamics of two vehicles, given in \eqref{eq:rel_dyn}. In general, the target state is specified to be some position $(\Pos_{x,r}, \Pos_{y,r})$ and velocity $(\Vel_{x,r}, \Vel_{y,r})$ relative to a reference vehicle. In the case of a vehicle joining a platoon that maintains a single file, the reference vehicle would be the platoon leader, the desired relative position would be a certain distance behind the leader, depending on how many other vehicles are already in the platoon; the desired relative velocity would be $(0,0)$ so that the formation can be kept.

For the reachability problem, we define the target set to be a small range of states around the target state $x_P = (\Pos_{x,r}, \Pos_{y,r}, \Vel_{x,r}, \Vel_{y,r})$:

\begin{equation}
\begin{aligned}
\mathcal{L}_P = \{x: |\pos_{x,r}-\Pos_{x,r}|\le r_{\pos_x}, |\vel_{x,r}-\Vel_{x,r}|\le r_{\vel_x}, \\
|\pos_{y,r} - \Pos_{y,r}| \le r_{\pos_y}, |\vel_{y,r} - \Vel_{y,r}|\le r_{\vel_y} \}
\end{aligned}
\end{equation}

The target set $\mathcal{L}_P$ is represented by the zero sublevel set of the implicit surface function $l_P(x)$, which specifies the terminal condition of the HJI PDE \eqref{eq:HJIPDE}. The zero sublevel set of the solution to \eqref{eq:HJIPDE}, $V_P(-T,x)$, gives us the set of relative states from which a quadrotor can reach the target in the relative coordinates within a duration of $T$. In the reachable set computation, we assume that the reference vehicle moves along the highway at constant speed, so that $u_j(t)$ = 0. The following is a suitable algorithm for a vehicle joining a platoon to follow the platoon leader:

\begin{enumerate}
\item Move towards $\bar{x}_P$ in a straight line, with some velocity, until $V_P(-T,x)\le 0$.
\item Apply the optimal control extracted from $V_P(-T,x)$ according to \eqref{eq:HJI_ctrl_syn} until $\mathcal{L}_P$ is reached.
\end{enumerate}

\subsubsection{Avoiding Collisions \label{sec:collision_ctrl}}
A vehicle can use a liveness controller described in the previous sections when it is not in any danger of collision with other vehicles. If the vehicle could potentially be involved in a collision within the next short period of time, it must switch to a safety controller. The safety controller is available in every mode, and executing the safety controller to perform an avoidance maneuver does not change a vehicle's mode. 

In the context of our platooning concept, we define an unsafe configuration as follows: a vehicle is either within a minimum separation distance $d$ to a reference vehicle in both the $x$ and $y$ directions, or is traveling with a speed above the speed limit $\vel_\text{max}$ in either of the $x$ and $y$ directions. To take this specification into account, we use the augmented relative dynamics given by \eqref{eq:rel_dyn_aug} for the reachability problem, and define the target set as follows:

\begin{equation}
\begin{aligned}
\mathcal{L}_S = \{x: &|\pos_{x,r}|, |\pos_{y,r}|\le d \vee |\vel_{x,i}| \ge \vel_\text{max} \vee |\vel_{y,i}| \ge \vel_\text{max} \}
\end{aligned}
\end{equation}

We can now define the implicit surface function $l_S(x)$ corresponding to $\mathcal{L}_S$, and solve the HJI PDE \eqref{eq:HJIPDE} using $l_S(x)$ as the terminal condition. As before, the zero sublevel set of the solution $V_S(t,x)$ specifies the reachable set $\mathcal{V}_S(t)$, which characterizes the states in the augmented relative coordinates, as defined in \eqref{eq:rel_dyn_aug}, from which $Q_i$ \textit{cannot} avoid $\mathcal{L}_S$ for a time period of $t$, if $Q_j$ uses the worst case control. To avoid collisions, $Q_i$ must apply the safety controller according to \eqref{eq:HJI_ctrl_syn} on the boundary of the reachable set in order to avoid going into the reachable set. The following algorithm wraps our safety controller around liveness controllers:

\begin{enumerate}
\item For a specified time horizon $t$, evaluate$V_S(-t,x_i-x_j)$ for all $j\in \mathcal{Q}(i)$.

$\mathcal{Q}(i)$ is the set of quadrotors with which quadrotor $i$ checks safety against.
\item Use the safety or liveness controller depending on the values $V_S(-t,x_i-x_j),j\in \mathcal{Q}(i)$: 

If $\exists j\in \mathcal{Q}(i),V_S(-t,x_i-x_j)\le 0$, then $Q_i,Q_j$ are in potential conflict, and $Q_i$ must use a safety controller; otherwise $Q_i$ uses a liveness controller.
\end{enumerate}

\subsection{Other Controllers \label{sec:other_ctrl}}
Reachability was used in Section \ref{sec:reach_ctrl} for the relatively complex maneuvers that require safety and liveness guarantees. For the simpler maneuvers of traveling along a highway and following a platoon, many well-known classical controllers suffice. For illustration, we use the simple controllers described below.

\subsubsection{Traveling along a highway} \label{sec:travel_hwy}
We use a model-predictive controller (MPC) for traveling along a highway; this controller allows the leader to travel along a highway at a pre-specified speed. Here, the goal is for a leader quadrotor to track a constant-altitude path, defined as a curve $\bar{p}(s)$ parametrized by $s\in[0,1]$ in $p=(p_x, p_y)$ space (position space), while maintaining a velocity $\bar{v}(s)$ that corresponds to constant speed in the direction of the highway. Assuming that the initial position on the highway, $s_0=s(t_0)$ is specified, such a controller can be obtained from the following optimization problem over the time horizon $[t_0, t_1]$:

\begin{equation}
\begin{aligned}
\text{minimize } & \int_{t_0}^{t_1} \big\{\| p(t)-\bar{p}(s(t)) \|_2 + \\ 
&\qquad \| v(t) - \bar{v}(s(t)) \|_2 + 1-s \big\} dt \\
\text{subject to } & \dot{x} = f(x,u) \text{ where } f \text{ is given in \eqref{eq:dyn}} \\
& |u_x|, |u_y| \le u_\text{max}, |v_x|, |v_y| \le v_\text{max} \\
& s(t_0) = s_0, \dot{s} \ge 0
\end{aligned}
\end{equation}

If we discretize time, and assume that $\bar{p}(\cdot)$ is linear, then the above optimization is convex, and can be quickly solved.

\subsubsection{Following a Platoon} \label{sec:follow_platoon}
Follower vehicles use a feedback control law tracking a nominal position and velocity in the platoon, with an additional feed-forward term given by the leader's acceleration input; here, for simplicity, we assume perfect communication between the leader and the follower vehicles. This following law enables smooth vehicle trajectories in the relative platoon frame, while allowing the platoon as a whole to perform agile maneuvers by transmitting the leader's acceleration command $u_{P_1}(t)$ to all vehicles.

The $i$-th member of the platoon, $Q_{P_i}$, is expected to track a relative position in the platoon $r^i = (r_x^i,r_y^i)$ with respect to the leader's position $p_{P_1}$, and the leader's velocity $v_{P_1}$ at all times. The resulting control law has the form:
\begin{equation}\label{eq:follow}
u^i(t) = k_p \big[p_{P_1}(t) + r^i(t) - p^i(t) \big] + k_v\big[v_{P_1}(t) - v^i(t)\big] + u_{P_1}(t)
\end{equation}
for some $k_p,k_v>0$. In particular, a simple rule for determining $r^i(t)$ in a single-file platoon is given for $Q_{P_i}$ as:
\begin{equation}\label{eq:nominal_pos}
r^i(t) = - (i-1) b \frac{v_{P_1}}{\|v_{P_1}\|_2}
\end{equation}
where $b$ is the spacing between vehicles along the platoon and $\frac{v_{P_1}}{\|v_{P_1}\|_2}$ is the platoon leader's direction of travel.
% !TEX root = main.tex
\subsection{Summary of Controllers}
We have introduced several reachability-based controllers, as well as some simple controllers. Pair-wise collision avoidance is guaranteed using the safety controller, described in Section \ref{sec:collision_ctrl}. As long as a vehicle is not in potential danger according to the safety reachable sets, it is free to use any other controller. All of these other controllers and their corresponding mode transitions are shown in Figure \ref{fig:modeControllers}.

The controller for getting to an absolute target state, described in Section \ref{sec:platooning}-\ref{sec:reach_ctrl}-\ref{sec:abs_target_ctrl}, is used whenever a vehicle needs to move onto a highway to become a platoon leader. This controller guarantees the success of the mode transitions shown in blue in Figure \ref{fig:modeControllers}.

The controller for getting to a relative target state, described in Section \ref{sec:platooning}-\ref{sec:reach_ctrl}-\ref{sec:rel_target_ctrl}, is used whenever a vehicle needs to join a platoon to become a follower. This controller guarantees the success of the mode transitions shown in green in Figure \ref{fig:modeControllers}.

For the simple maneuvers of traveling along a highway or following a platoon, many simple controllers such as the ones suggested in Section \ref{sec:platooning}-\ref{sec:other_ctrl} can be used. These controllers keep the vehicles in either the Leader or the Follower mode. Alternatively, additional controllers can be designed for exiting the highway, although these are not considered in this paper. All of these non-reachability-based controllers are shown in gray in Figure \ref{fig:modeControllers}.

\begin{figure}
	\centering
	\includegraphics[width=0.5\textwidth]{"fig/modeControllers"}
	\caption{}
	\label{fig:modeControllers}
\end{figure}
% !TEX root = main.tex
\subsection{Safety Analysis}
Under normal operations in a single platoon, each follower vehicle $\veh{i},i=P_2,\ldots,P_{M-1}$ in a platoon checks whether it is in the safety BRS with respect to $\veh{P_{i-1}}$ and $\veh{P_{i+1}}$. So $\vehSCS{i} = \{P_{i+1}, P_{i-1}\}$ for $i=P_2,\ldots,P_{N-1}$. Assuming there are no nearby vehicles outside of the platoon, the platoon leader $\veh{P_1}$ checks safety against $\veh{P_2}$, and the platoon trailer $Q_{P_N}$ checks safety against $Q_{P_{N-1}}$. So $\vehSCS{P_1}=\{P_2\}, \vehSCS{P_N}=\{P_{N-1}\}$. When all vehicles are using goal satisfaction controllers to perform their allowed maneuvers, in most situations no pair of vehicles should be in an unsafe configuration. However, occasionally a vehicle $\veh{k}$ may behave unexpectedly due to faults or malfunctions, in which case it may come into an unsafe configuration with another vehicle.

With our choice of $\vehSCS{i}$ and the assumption that the platoon is in a single-file formation, some vehicle $\veh{i}$ would get near the safety BRS with respect to $\veh{k}$, where $\veh{k}$ is likely to be the vehicle in front or behind of $\veh{i}$. In this case, a ``safety breach" occurs. Our synthesis of the safety controller guarantees that between every pair of vehicles $\veh{i},\veh{k}$, as long as $V_S(-t,x_i- x_k)>0$, $\exists u_i$ to keep $\veh{i}$ from colliding with $\veh{k}$ for a desired time horizon $t$, despite the worst case (an adversarial) control from $\veh{k}$. Therefore, as long as the number of ``safety breaches" is at most one for $\veh{i}$, $Q_i$ can simply use the optimal control to avoid $\veh{k}$ and avoid collision for the time horizon of $t$. Under the assumption that vehicles are able to exit the current altitude range within a duration of $\td$, if we choose $t=\td$, the safety breach would always end before any collision can occur. 

Within a duration of $\td$, there is a small chance that additional safety breaches may occur. However, as long as the total number of safety breaches does not exceed the number of affected vehicles, collision avoidance of all the vehicles can be guaranteed for the duration $\td$. However, as our simulation results show, placing vehicles in single-file platoons makes the likelihood of multiple safety breaches low during the presence of one intruder vehicle. 

In the event that multiple safety breaches occur for some of the vehicles due to a malfunctioning vehicle within the platoon or intruding vehicles outside of the platoon, vehicles that are causing safety breaches must exit the highway altitude range in order to avoid collisions. Every extra altitude range reduces the number of simultaneous safety breaches by $1$, so $K$ simultaneous safety breaches can be resolved using $K-1$ different altitude ranges. The general process and details of the complete picture of multi-altitude collision avoidance is part of our future work. 

The concept of platooning can be coupled with any collision avoidance algorithm that provides safety guarantees. In this paper, we have only proposed the simplest reachability-based collision avoidance scheme. Existing collision avoidance algorithms such as \cite{Bansal16} and \cite{Chen16} have the potential to provide safety guarantees for many vehicles in order to resolve multiple safety breaches at once. Coupling the platooning concept with the more advanced collision avoidance methods that provide guarantees for a larger number of vehicles would reduce the risk of multiple safety breaches.

Given that vehicles within a platoon are safe with respect to each other, each platoon can be treated as a single vehicle, and perform collision avoidance with other platoons. By treating each platoon as a single unit, we can reduce the number of individual vehicles that need to check for safety against each other, reducing overall computation burden.
% !TEX root = main.tex
% 82 124 170 231
\subsection{Numerical Simulations}

In this section, we consider several situations that quadrotors in a platoon on an air highway may commonly encounter, and show via simulations the behaviors that emerge from the controllers we defined in Sections \ref{sec:liveness} and \ref{sec:safety}.

\subsubsection{Forming a Platoon}
We first consider the scenario in which some quadrotors are trying to merge onto an initially unoccupied highway. In order to do this, each quadrotor first checks for safety with respect to the other quadrotors, and uses the safety controller if necessary, according to Section \ref{sec:safety}. Otherwise, the quadrotor uses the liveness controller described in Section \ref{sec:liveness}. 

For the simulation example, the highway is specified by the line $p_y = 0.5p_x$, the point of entry on the highway is chosen to be $(\bar{p}_x, \bar{p}_y) = (4,2)$, and the velocity on the highway is chosen to be ($\bar{v}_x, \bar{v}_y) = \frac{\bar{v}}{\sqrt{0.5^2 + 1^2}} (0.5, 1)$. The velocity simply states that the quadrotors must travel at a speed $\bar{v}=3$ along the direction of the highway. This forms the target state $\bar{x}_H=(\bar{p}_x, \bar{v}_x, \bar{p}_y, \bar{v}_y)$, from which we define the target set $\mathcal{L}_H$ as in Section \ref{subsec:highway_merge}.

The first quadrotor that completes merging onto the empty highway creates a platoon and becomes its leader, while subsequent quadrotors form a platoon behind the leader in a pre-specified order according to the process described in Section \ref{subsec:platoon_merge}. Here, we choose $(\bar{p}_{x,r}, \bar{p}_{y,r})$ to be a distance $b$ behind the last quadrotor in the platoon, and $(\bar{v}_{x,r}, \bar{v}_{y,r}) = (0,0)$. This gives us the target set $\mathcal{L}_P$ that we need.

Figures \ref{fig:normal2} and \ref{fig:normal5} show the simulation results. Since the liveness reachable sets are in 4D and the safety reachable sets are in 6D, we compute and plot their 2D slices based on the quadrotors' velocities and relative velocities. 

Figure \ref{fig:normal2} illustrates the use of liveness and safety reachable sets using just two quadrotors to reduce visual clutter. The first quadrotor $Q_1$ (red disk) first travels in a straight line towards the highway merging point $\bar{x}$ (red circle) at $t=1.5$, because it is not yet in the liveness reachable set for merging onto the highway (red dotted boundary). When it is within the liveness reachable set boundary at $t=2.8$, it is ``locked-in" to the target state $\bar{x}_H$, and follows the optimal control in \eqref{eq:HJB_ctrl_syn} to $\bar{x}_H$. During the entire time, the $Q_1$ checks whether it may collide with $Q_2$ within a time horizon of $t_\text{external}$; we chose $t_\text{external}=3>t_\text{internal}$. However, since $Q_1$ never goes into the boundary of the safety reachable set (red dashed boundary), it uses the liveness controller the entire time.

After $Q_1$ has reached $\bar{x}_H$, it forms a platoon, becomes the platoon leader, and continues to travel along the highway. $Q_2$ (blue disk), at $t=7$, begins joining the platoon behind $Q_1$, by moving towards the target $\bar{x}_P$ relative to the position of $Q_1$. Note that $\bar{x}_P$ moves with $Q_1$ as $\bar{x}_P$ is defined in terms of the relative states of the two quadrotors. When $Q_2$ moves inside the liveness reachable set boundary for joining the platoon (blue dotted boundary), it is ``locked-in" to the target relative state $\bar{x}_P$, and begins following the optimal control in \eqref{eq:HJI_ctrl_syn} towards the target as long as it stays out of the safety reachable set (blue dashed boundary).

Figure \ref{fig:normal5} shows the behavior of all 5 quadrotors which eventually form a platoon and travel along the highway together. The liveness controllers allow the quadrotors to optimally and smoothly enter the highway and join platoons, while the safety controllers prevent collisions from occurring.

% 40 60 110 210
\begin{figure} \label{fig:fp}
    \centering
    \begin{subfigure}{0.23\textwidth} \label{subfig:fp_40}
        \includegraphics[width=\textwidth]{fig/fp_40}
        \caption{40}
    \end{subfigure}
    \begin{subfigure}{0.23\textwidth} \label{subfig:fp_60}
        \includegraphics[width=\textwidth]{fig/fp_60}
        \caption{60}
    \end{subfigure}

    \begin{subfigure}{0.23\textwidth} \label{subfig:fp_110}
        \includegraphics[width=\textwidth]{fig/fp_110}
        \caption{110}
    \end{subfigure}
    \begin{subfigure}{0.23\textwidth} \label{subfig:fp_210}
        \includegraphics[width=\textwidth]{fig/fp_210}
        \caption{210}
    \end{subfigure}   
    \caption{form platoons}    
\end{figure}

\subsubsection{Intruder Vehicle}
We now consider the scenario in which a platoon of quadrotors encounters an intruder vehicle. To avoid collision, each quadrotor checks for safety with respect to the intruder and any quadrotor in front and behind in the platoon. If necessary, the quadrotor uses the safety controller, otherwise it uses the appropriate liveness controller, depending on whether it is a leader or follower.

Figure \ref{fig:intruder1} shows the simulation result. At $t=0$, a platoon of 4 quadrotors, $Q_{P_i},i=1,\ldots,4$ with $P_i = i$, travels along the highway defined by the line $p_y = p_x$. An intruder vehicle $Q_0$ (red disk) starts from position $(p_x, p_y) = (40,30)$ and heads toward bottom-left of the grid. As before, platoon members incur a safety breach if $V(-t_\text{internal},x_{P_i}-x_{P_j})\le 0$. However, with respect to $Q_0$, platoon members incur a safety breach if $V(-t_\text{external}, x_{P_i}-x_0) \le 0$ with $t_\text{external}>t_\text{internal}$. 

The platoon leader $Q_{P_1}$'s (black disk) safety is unaffected by the intruder, thus it simply follows its original path on the highway using the liveness controller described in Section \ref{sec:travel_hwy}. Followers $Q_{P_2}$ (blue disk), $Q_{P_3}$ (green disk) and $Q_{P_4}$ (pink disk), on the other hand, must use the safety controller in order to avoid collision with the intruder ($t=3.3,6.2$). This causes their paths to deviate off the highway. Once each quadrotor is safe relative to the intruder, they merge back onto the highway, join the original platoon and continue traveling along it ($t=12.4$).

Figure \ref{fig:intruder2} illustrates the use of safety reachable sets in this scenario using only $Q_{P_2}$ as an example. The safety reachable sets of $Q_{P_2}$ with respect to the intruder $Q_0$ (red dashed line), $Q_{P_1}$ (black dashed line) and $Q_{P_3}$ (green dashed line) are shown. With respect to $Q_0$, $Q_{P_2}$'s safety is considered to be breached if $V_S(-t_\text{external},x_{P_2}-x_0) \le 0$. To avoid possible collision with the intruder, $Q_{P_2}$ must remain outside the safety reachable set with respect to the intruder. The same applies to collision avoidance with $Q_{P_1}$ and $Q_{P_3}$. %\textcolor{red}{\sout{Note that safety reachable sets with $Q_{P_1}$ and $Q_{P_3}$ are much smaller than that with the intruder. This is because under normal conditions, vehicles in the same platoon have near-zero relative velocity. For this reason, they may only check possible collisions within the next 1.5 seconds, instead of 3 seconds for vehicles outside of the platoon. This allows vehicles to travel closer together within a platoon and thus increasing throughput on the air highway.}} \textcolor{blue}{The 1.5 seconds is because we made the assumption that malfunctioning vehicles will change altitude within 1.5 seconds.}

Initially, $Q_{P_2}$ ($P_2=2$) is a follower and is outside all 3 safety reachable sets. Hence it is allowed to use the liveness controller to follow the platoon as described in Section \ref{sec:follow_platoon}. At time $t=0.6$, $Q_2$ comes to the boundary of the safety set with respect to the intruder and therefore must apply the safety control law to avoid potential future collision. Thus it splits the original platoon and becomes the leader of a new platoon consisting of itself, $Q_3$ and $Q_4$. $Q_2$ keeps using the safety controller until it is safe with respect to the intruder again at $t=3$, since if it tried to merge back to the highway before this time, it would enter the safety set with respect to the intruder and lose the $t_\text{external}$ safety guarantee. After $t=3$, $Q_2$ is safe to use the liveness controller again to merge back onto the highway and join the original platoon as a follower. Note that during the entire time, $Q_2$ maintains safety against the intruder, $Q_1$ and $Q_3$ by always staying outside of all three safety reachable sets.

% 100 120 140 200
\begin{figure} \label{fig:in}
    \centering
    \begin{subfigure}{0.23\textwidth} \label{subfig:in_100}
        \includegraphics[width=\textwidth]{fig/in_100}
        \caption{100}
    \end{subfigure}
    \begin{subfigure}{0.23\textwidth} \label{subfig:in_120}
        \includegraphics[width=\textwidth]{fig/in_120}
        \caption{120}
    \end{subfigure}

    \begin{subfigure}{0.23\textwidth} \label{subfig:in_140}
        \includegraphics[width=\textwidth]{fig/in_140}
        \caption{140}
    \end{subfigure}
    \begin{subfigure}{0.23\textwidth} \label{subfig:in_200}
        \includegraphics[width=\textwidth]{fig/in_200}
        \caption{200}
    \end{subfigure}   
    \caption{intruder}    
\end{figure}

\subsubsection{Changing highways}

\begin{figure} \label{fig:ch}
    \centering
    \begin{subfigure}{0.23\textwidth} \label{subfig:ch_83}
        \includegraphics[width=\textwidth]{fig/ch_83}
        \caption{83}
    \end{subfigure}
    \begin{subfigure}{0.23\textwidth} \label{subfig:ch_124}
        \includegraphics[width=\textwidth]{fig/ch_124}
        \caption{124}
    \end{subfigure}

    \begin{subfigure}{0.23\textwidth} \label{subfig:ch_170}
        \includegraphics[width=\textwidth]{fig/ch_170}
        \caption{170}
    \end{subfigure}
    \begin{subfigure}{0.23\textwidth} \label{subfig:ch_231}
        \includegraphics[width=\textwidth]{fig/ch_231}
        \caption{231}
    \end{subfigure}   
    \caption{change highways}    
\end{figure}

% !TEX root = main.tex
\section{Conclusions}
We considered single-file platoons of UAVs modeled by hybrid systems traveling along air highways. Using HJ reachability, we proposed liveness controllers for merging onto highways and merging into existing platoons, and wrapped a safety controller around liveness controllers to ensure no collision between the UAVs can occur. Under the assumption that faulty vehicles can descend after a pre-specified duration, our safety controller guarantees that no collisions will occur in a single altitude level as long as at most one safety breach occurs for each vehicle in the platoon. Additional safety breaches can be handled by multiple altitude ranges in the airspace. Our simulations show that by putting vehicles into single-file platoons, the likelihood of having multiple safety breaches is low, and conflicts involving a single malfunctioning UAV or intruder can be resolved in a single altitude level.

Immediate future work includes exploring different vehicle models, investigating algorithms for resolving multiple safety breaches within the same altitude, and more broadly related problems such as airspace structure and air highway placement.


%\section*{Appendix}

\section*{Acknowledgments}
This work is supported in part by NSF under CPS:ActionWebs (CNS-0931843) and CPS:FORCES (CNS1239166), by NASA under grants NNX12AR18A and UCSCMCA-14-022 (UARC), by ONR under grants N00014-12-1-0609, N000141310341 (Embedded Humans MURI), and MIT\_5710002646 (SMARTS MURI), and by AFOSR under grants UPenn-FA9550-10-1-0567 (CHASE MURI) and the SURE project.

\bibliographystyle{aiaa}
\bibliography{references}
\end{document}
