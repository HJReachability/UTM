% !TEX root = main.tex
\subsection{Results}
To illustrate our air highway placement proposal, we used the San Francisco Bay Area as an example, and classified each point on the map into four different regions: ``regions around airports'', ``highly populated cities'', ``water'', and ``other''. Each region has an associated cost, reflecting the desirability of flying a vehicle over an area in the region. In general, these costs can be arbitrary and determined by government regulation agencies. For illustration purposes, we assumed the following categories and costs:

\begin{itemize}
\item Region around airports: $\cost{airports}=b$,
\item Cities: $\cost{cities}=1$,
\item Water: $\cost{water}=b^{-2}$,
\item Other: $\cost{other}=b^{-1}$.
\end{itemize}

This assumption assigns costs in descending order to the categories ``regions around airports'', ``cities'', ``other'', and ``water''. Flying a UAV in each category is more costly by a factor of $b$ compared to the next most important category. The factor $b>1$ is a tuning parameter that we adjusted to vary the relative importance of the different categories, and we used $b=4$ in the figures below.

Figure \ref{fig:airHighway_results} shows the San Francisco Bay Area (geographic) map, cost map, cost-minimizing paths, and contours of the value function $V$. The region enclosed by the black boundary represents ``region around airports'', which have the highest cost. The dark blue, yellow, and light blue regions represent the ``cities'', the ``water'', and the ``other''' categories, respectively. We assumed that the origin corresponds to the city ``Concord'', and chose a number of other major cities as destinations.

A couple of important observations can be made here. First, the cost-minimizing paths to the various destinations in general overlap, and only split up when they are very close to entering their destination cities. This is intuitively desirable because the number of air highways is kept low. In addition, the cost of flying in the airspace according to the cost map is minimized. Secondly, the contours, which correspond to level curves of the value function, have a spacing corresponding to the cost map. This provides insight into the placement of air highways to destinations that were not shown in this example.

\begin{figure}
	\centering
	\includegraphics[width=0.95\textwidth]{"fig/airHighway_results"}
	\caption{Cost-minimizing paths computed by the Fast Marching Method based on the assumed cost map of the San Francisco Bay Area}
	\label{fig:airHighway_results}
\end{figure}

Figure \ref{fig:airHighway_sparse} shows the result of converting the cost-minimizing paths to a small number of waypoints. The left plot shows the waypoints, interpreted as the start and end points of air highways, over a white background for clarity. The right plot shows these air highways over the map of the Bay Area. Note that we could have gone further to merge some of the overlapping highways. However, the purpose of this section is to illustrate the natural occurrence of air highways from cost-minimizing paths; post-processing of the cost-minimizing paths is not our focus.

\begin{figure}
	\centering
	\includegraphics[width=0.95\textwidth]{"fig/airHighway_sparse"}
	\caption{Results of conversion from cost-minimizing paths to highway way points.}
	\label{fig:airHighway_sparse}
\end{figure}

\subsection{Real-Time Highway Location Updates}
\MCnote{New section}Since \eqref{eq:eikonal} can be solved in approximately 1 second, the air highway placement process can be redone in real-time if the cost map changes. This flexibility can be useful in any situation in which unforeseen circumstances could cause a change in the cost of a particular region of the airspace. For example, accidents or disaster response vehicles may result in an area temporarily having a high cost. On the other hand, depending on for instance the time of day, it may be most desirable to fly in different regions of the airspace, resulting in those regions temporarily having a low cost.