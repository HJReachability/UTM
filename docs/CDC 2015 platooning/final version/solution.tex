% !TEX root = platooning.tex
\section{Hamilton-Jacobi Reachability  \label{sec:HJI}}
\subsection{General Framework}
Consider a differential game between two players described by the system
\begin{equation} \label{eq:dyn}
\dot{x} = f(x, u_1, u_2), \text{for almost every }t\in [-T,0]
\end{equation}

\noindent where $x\in\mathbb{R}^n$ is the system state, $u_1\in \mathcal{U}_1$ is the control of Player 1, and $u_2\in\mathcal{U}_2$ is the control of Player 2. We assume $f:\mathbb{R}^n\times \mathcal{U}_1 \times \mathcal{U}_2 \rightarrow \mathbb{R}^n$ is uniformly continuous, bounded, and Lipschitz continuous in $x$ for fixed $u_1,u_2$, and the control functions $u_1(\cdot)\in\mathbb{U}_1,u_2(\cdot)\in\mathbb{U}_2$ are drawn from the set of measurable functions. Player 2 is allowed to use nonanticipative strategies \cite{Evans84,Varaiya67} $\gamma$, defined by

\begin{equation}
\begin{aligned}
\gamma &\in \Gamma := \{\mathcal{N}: \mathbb{U}_1 \rightarrow \mathbb{U}_2 \mid  u_1(r) = \hat{u}_1(r) \\
&\text{for almost every } r\in[t,s] \Rightarrow \mathcal{N}[u_1](r) \\
&= \mathcal{N}[\hat{u}_1](r) \text{ for almost every } r\in[t,s]\}
\end{aligned}
\end{equation}

In our differential game, the goal of Player 2 is to drive the system into some target set $\mathcal{L}$, and the goal of Player 1 is to drive the system away from it. The set $\mathcal{L}$ is represented as the zero sublevel set of a bounded, Lipschitz continuous function $l:\mathbb{R}^n\rightarrow\mathbb{R}$. We call $l(\cdot)$ the \textit{implicit surface function} representing the set $\mathcal{L}=\{x\in\mathbb{R}^n \mid l(x)\le 0\}$.

Given the dynamics \eqref{eq:dyn} and the target set $\mathcal{L}$, we would like to compute the backwards reachable set, $\mathcal{V}(t)$:

\begin{equation}
\begin{aligned}
\mathcal{V}(t) &:= \{x\in\mathbb{R}^n \mid \exists \gamma\in\Gamma \text{ such that } \forall u_1(\cdot)\in\mathbb{U}_1, \\
&\exists s \in [t,0], \xi_f(s; t, x, u_1(\cdot), \gamma[u_1](\cdot)) \in \mathcal{L} \}
\end{aligned}
\end{equation}
where $\xi_f$ is the trajectory of the system satisfying initial conditions $\xi_f(t; x, t, u_1(\cdot), u_2(\cdot))=x$ and the following differential equation almost everywhere on $[-t, 0]$:
\begin{equation}
\begin{aligned}
\frac{d}{ds}&\xi_f(s; x, t, u_1(\cdot), u_2(\cdot)) \\
&= f(\xi_f(s; x, t, u_1(\cdot), u_2(\cdot)), u_1(s), u_2(s))
\end{aligned}
\end{equation}

For this paper, we use the HJ formulation in \cite{Mitchell05}, which has shown that the backwards reachable set $\mathcal{V}(t)$ can be obtained as the zero sublevel set of the viscosity solution \cite{Crandall84} $V(t,x)$ of the following terminal value Hamilton-Jacobi-Isaacs (HJI) PDE:

\begin{equation} \label{eq:HJIPDE}
\begin{aligned}
D_t &V +\min \{0, \max_{u_1\in\mathcal{U}_1} \min_{u_2\in\mathcal{U}_2} D_x V \cdot f(x,u_1,u_2) \} = 0, \\
&V(0,x) = l(x)
\end{aligned}
\end{equation}

\noindent from which we obtain $\mathcal{V}(t) = \{x\in\mathbb{R}^n \mid V(t,x)\le 0\}$. From the solution $V(t,x)$, we can also obtain the optimal controls for both players via the following:

\begin{equation} \label{eq:HJI_ctrl_syn}
\begin{aligned}
u_1^*(t,x) &= \arg \max_{u_1\in\mathcal{U}_1} \min_{u_2\in\mathcal U_2} D_x V(t,x) \cdot f(x,u_1,u_2)\\
u_2^*(t,x) &= \arg \min_{u_2\in\mathcal{U}_2} D_x V(t,x) \cdot f(x,u_1^*,u_2)
\end{aligned}
\end{equation}

In the special case where there is only one player (Player 2 does not exist), we obtain an optimal control problem for a system with dynamics

\begin{equation} \label{eq:dyn_d}
\dot{x} = f(x, u), t\in [-T,0], u\in\mathcal U.
\end{equation}

The reachable set in this case would be given by the Hamilton-Jacobi-Bellman (HJB) PDE

\begin{equation} \label{eq:HJBPDE}
\begin{aligned}
D_t V(t,x) + \min \{0, \min_{u\in\mathcal{U}} D_x V(t,x) \cdot f(x,u)\} &= 0 \\
V(0,x) = l(x)&
\end{aligned}
\end{equation}

\noindent where the optimal control is given by

\begin{equation} \label{eq:HJB_ctrl_syn}
u^*(t,x) = \arg \min_{u\in\mathcal{U}} D_x V(t,x) \cdot f(x,u)
\end{equation}

For our application, we will use a several decoupled system models and utilize the decoupled HJ formulation in \cite{Chen15}, which enables real time 4D reachable set computations and tractable 6D reachable set computations.

\section{Liveness Controllers \label{sec:liveness}}
\subsection{Merging onto a Highway \label{subsec:highway_merge}}
We model the merging of a vehicle onto an air highway as a path planning problem, where we specify a target position and velocity along the highway. Thus, a vehicle would aim to drive the system \eqref{eq:dyn} to a state $\bar{x}_H=(\bar{p}_x, \bar{v}_x, \bar{p}_y, \bar{v}_y)$, or a small range of states defined by the set

\begin{equation}
\begin{aligned}
\mathcal{L}_H = \{x: |p_x-\bar{p}_x|\le r_{p_x}, |v_x-\bar{v}_x|\le r_{v_x}, \\
|p_y - \bar{p}_y| \le r_{p_y}, |v_y - \bar{v}_y|\le r_{v_y} \}.
\end{aligned}
\end{equation}

In this reachability problem, $\mathcal{L}_H$ is the target set, represented by the zero sublevel set of the function $l_H(x)$, which specifies the terminal condition of the HJB PDE to be solved. The solution we obtain, $V_H(t,x)$, is the implicit surface function representing the reachable set $\mathcal V_H(t)$; $V_H(-T,x)\le 0$, then, specifies the reachable set $\mathcal{V}_H(T)$, the set of states from which the system can be driven to the target $\mathcal{L}_H$ within a duration of $T$. This gives the algorithm for merging onto the highway:

\begin{enumerate}
\item Move towards $\bar{x}_H$ in a straight line until $V_H(-T,x)\le 0$. This simple controller is chosen heuristically.
\item Apply the optimal control extracted from $V_H(-T,x)$ according to \eqref{eq:HJB_ctrl_syn} until $\mathcal{L}_H$ is reached.
\end{enumerate}

\subsection{Merging into a Platoon \label{subsec:platoon_merge}}
We again pose the merging of a vehicle into a platoon on an air highway as a reachability problem. Here, we would like quadrotor $Q_i$ to merge onto the highway and follow another vehicle $Q_j$ in a platoon. Thus, we would like to drive the system given by \eqref{eq:rel_dyn_aug} to a specific $\bar{x}_P = (\bar{p}_{x,r}, \bar{v}_{x,r}, \bar{p}_{y,r}, \bar{v}_{y,r})$, or a small range of relative states defined by the set

\begin{equation}
\begin{aligned}
\mathcal{L}_P = \{x: |p_{x,r}-\bar{p}_{x,r}|\le r_{p_x}, |v_{x,r}-\bar{v}_{x,r}|\le r_{v_x}, \\
|p_{y,r} - \bar{p}_{y,r}| \le r_{p_y}, |v_{y,r} - \bar{v}_{y,r}|\le r_{v_y} \}
\end{aligned}
\end{equation}

The target set $\mathcal{L}_P$ is represented by the implicit surface function $l_P(x)$, which specifies the terminal condition of the HJI PDE \eqref{eq:HJIPDE}. The zero sublevel set of the solution to \eqref{eq:HJIPDE}, $V_P(-T,x)$, gives us the set of relative states from which $Q_i$ can reach the target and join the platoon following $Q_j$ within a duration of $T$. We assume that $Q_j$ moves along the highway at constant speed, so that $u_j(t)$ = 0. The following is a suitable algorithm for a vehicle merging onto a highway and joining a platoon to follow $Q_j$:

\begin{enumerate}
\item Move towards $\bar{x}_P$ in a straight line until $V_P(-T,x)\le 0$.
\item Apply the optimal control extracted from $V_P(-T,x)$ according to \eqref{eq:HJI_ctrl_syn} until $\mathcal{L}_P$ is reached.
\end{enumerate}

\subsection{Other Quadrotor Maneuvers}
For the simpler maneuvers of traveling along a highway and following a platoon, we resort to simpler controllers described below.

\subsubsection{Traveling along a highway} \label{sec:travel_hwy}
We use a model-predictive controller (MPC) for traveling along a highway at a pre-specified speed. Here, a leader quadrotor tracks a constant-altitude path, defined as a curve $\bar{p}(s)$ parametrized by $s\in[0,1]$ in $p=(p_x, p_y)$ space (position space), while maintaining a velocity $\bar{v}(s)$ that corresponds to constant speed in the direction of the highway.

\subsubsection{Following a Platoon} \label{sec:follow_platoon}
Follower vehicles use a feedback control law tracking a nominal position and velocity in the platoon, with an additional feed-forward term given by the leader's acceleration input.

The $i$-th member of the platoon, $Q_{P_i}$, is expected to track a relative position in the platoon $r^i = (r_x^i,r_y^i)$ with respect to the leader's position $p_{P_1}$, and the leader's velocity $v_{P_1}$ at all times. The resulting control law has the form:
\begin{equation}\label{eq:follow}
u^i(t) = k_p \big[p_{P_1}(t) + r^i(t) - p^i(t) \big] + k_v\big[v_{P_1}(t) - v^i(t)\big] + u_{P_1}(t)
\end{equation}
for some $k_p,k_v>0$. A simple rule for determining $r^i(t)$ in a single-file platoon is given for $Q_{P_i}$ as:
\begin{equation}\label{eq:nominal_pos}
r^i(t) = - (i-1) b \frac{v_{P_1}}{\|v_{P_1}\|_2}
\end{equation}
where $b$ is the spacing between vehicles along the platoon. and $\frac{v_{P_1}}{\|v_{P_1}\|_2}$ is the platoon leader's direction of travel.

\section{Safety Controllers \label{sec:safety}}
\subsection{Wrapping Reachability Around Existing Controllers}
A quadrotor can use a liveness controller when it is not in any danger of collision with other quadrotors or obstacles. If the quadrotor could potentially be involved in a collision within the next short period of time, it must switch to a safety controller. In this section, we will demonstrate how HJ reachability can be used to both detect imminent danger and synthesize a controller that guarantees safety within a specified time horizon. For our safety analysis, we will use the model in \eqref{eq:rel_dyn_aug}.

We begin by defining the target set $\mathcal{L}_S$, which characterizes configurations in relative coordinates for which vehicles $Q_i,Q_j$ are considered to be in collision:

\begin{equation}
\begin{aligned}
\mathcal{L}_S = \{x: &|p_{x,r}|, |p_{y,r}|\le d \vee |v_{x,i}| \ge v_\text{max} \vee |v_{y,i}| \ge v_\text{max} \}
\end{aligned}
\end{equation}

With this definition, $Q_i$ is considered to be unsafe if $Q_i$ and $Q_j$ are within a distance $d$ in both $x$ and $y$ directions simultaneously, or if $Q_i$ has exceeded some maximum speed $v_\text{max}$ in either $x$ or $y$ direction. For illustration purposes, we choose $d=2$ meters, and $v_\text{max}= 5$ m/s.

We can now define the implicit surface function $l_S(x)$ corresponding to $\mathcal{L}_S$, and solve the HJI PDE \eqref{eq:HJIPDE} using $l_S(x)$ as the terminal condition. As before, the zero sublevel set of the solution $V_S(t,x)$ specifies the reachable set $\mathcal{V}_S(t)$, which characterizes the states in the augmented relative coordinates, as defined in \eqref{eq:rel_dyn_aug}, from which $Q_i$ \textit{cannot} avoid $\mathcal{L}_S$ for a time period of $t$, if $Q_j$ uses the worst case control. To avoid collisions, $Q_i$ must apply the safety controller according to \eqref{eq:HJI_ctrl_syn} on the boundary of the reachable set in order to avoid going into the reachable set. The following algorithm wraps our safety controller around liveness controllers:

\begin{enumerate}
\item For a specified time horizon $t$, evaluate$V_S(-t,x_i-x_j)$ for all $j\in \mathcal{Q}(i)$.

$\mathcal{Q}(i)$ is the set of quadrotors with which quadrotor $i$ checks safety against. We discuss $\mathcal{Q}(i)$ in Section \ref{subsec:safety_guarantees}.

\item Use the safety or liveness controller depending on the values $V_S(-t,x_i-x_j),j\in \mathcal{Q}(i)$: 

If $\exists j\in \mathcal{Q}(i),V_S(-t,x_i-x_j)\le 0$, then $Q_i,Q_j$ are in potential conflict, and $Q_i$ must use a safety controller; otherwise $Q_i$ uses a liveness controller.
\end{enumerate}

\subsection{Platoon Safety Guarantees \label{subsec:safety_guarantees}}
Under normal operations in a single platoon, each follower quadrotor $Q_{P_i},i>1$ checks whether it is in the safety reachable set with respect to $Q_{P_{i-1}}$ and $Q_{P_{i+1}}$. So $\mathcal{Q}(i) = \{P_{i+1}, P_{i-1}\}$ for $i=P_2,\ldots,P_{N-1}$. Assuming there are no nearby quadrotors outside of the platoon, the platoon leader $Q_{P_1}$ checks safety against $Q_{P_2}$, and the platoon trailer $Q_{P_N}$ checks safety against $Q_{P_{N-1}}$. So $\mathcal{Q}(P_1)=\{P_2\}, \mathcal{Q}(P_N)=\{P_{N-1}\}$. No pair of quadrotors should be in an unsafe configuration if the liveness controllers are well-designed. Occasionally, a quadrotor $Q_k$ may behave unexpectedly due to faults, which may lead to an unsafe configuration.

With our choice of $\mathcal{Q}(i)$ and the assumption that the platoon is in a single-file formation, some quadrotor $Q_i$ would get into an unsafe configuration with $Q_k$, where $Q_k$ is likely to be the quadrotor in front or behind of $Q_i$. In this case, a ``safety breach" occurs. Our synthesis of the safety controller guarantees that between every pair of quadrotors $Q_i,Q_k$, as long as $V_S(-t,x_i- x_k)>0$, $\exists u_i$ to keep $Q_i$ from colliding with $Q_k$ for a desired time horizon $t$, despite the worst case (an adversarial) control from $Q_k$. Therefore, as long as the number of ``safety breaches" is at most one, $Q_i$ can simply use the optimal control $u_i$ to avoid collision with $Q_k$ for the time horizon of $t$. Since by assumption, vehicles in platoons are able to exit the current altitude range within a duration of $t_\text{internal}$, if we choose $t=t_\text{internal}$, the safety breach would always end before any collision can occur. 

Within a duration of $t_\text{internal}$, there is a small chance that additional safety breaches may occur. However, as long as the total number of safety breaches does not exceed the number of affected quadrotors, collision avoidance of all the quadrotors can be guaranteed for the duration $t_\text{internal}$. However, as our simulation results show, putting quadrotors in single-file platoons makes the likelihood of multiple safety breaches low during a quadrotor malfunction and during the presence of one intruder vehicle. 

In the event that multiple safety breaches occur for some of the quadrotors due to a malfunctioning quadrotor within the platoon or an intruding quadrotor outside of the platoon, those quadrotors with more than one safety breach still have the option of exiting the highway altitude range in order to avoid collisions. Every extra altitude range reduces the number of simultaneous safety breaches by $1$, so $K$ simultaneous safety breaches can be resolved using $K-1$ different altitude ranges. 

Given that quadrotors within a platoon are safe with respect to each other, each platoon can be treated as a single vehicle, and perform collision avoidance with other platoons. By treating each platoon as a single unit, we reduce the number of individual quadrotors that need to check for safety against each other, reducing overall computation burden.