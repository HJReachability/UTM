% !TEX root = main.tex
\section{Introduction}
Unmanned aerial vehicles (UAVs) have in the past been mainly used for military operations \cite{Tice91}; however, recently there has been an immense surge of interest in using UAVs for civil applications. Through projects such as Amazon Prime Air \cite{PrimeAir} and Google Project Wing \cite{ProjectWing}, companies are looking to send UAVs into the airspace to not only deliver commercial packages, but also for important tasks such as aerial surveillance, emergency supply delivery, videography, and search and rescue \cite{Kopardekar16}. In the future, the use of UAVs \MCnote{is likely to become more and more prevalent.}

As a rough estimate, suppose in a city of 2 million people, each person requests a drone delivery every 2 months on average and each delivery requires a 30-minute trip for a UAV. This would equate to thousands of UAVs simultaneously in the air just from package delivery services. Applications of UAVs extend beyond package delivery; they can also be used, for example, to provide supplies or to respond to disasters in areas that are difficult to reach but require prompt response \cite{Debusk10,Tornado16}. As a result, government agencies such as the Federal Aviation Administration (FAA) and National Aeronautics and Space Administration (NASA) are also investigating unmanned aerial systems (UAS) traffic management (UTM) in order to prevent collisions among potentially numerous UAVs \cite{Kopardekar16, FAA13, NASA16}. 

Optimal control and game theory present powerful tools for providing safety and goal satisfaction guarantees to controlled dynamical systems under bounded disturbances, and various formulations \cite{Bokanowski10, Mitchell05, Barron89} have been successfully used to analyze problems involving small numbers of vehicles \cite{Fisac15, Chen14, Chen17, Ding08}. These formulations are based on Hamilton-Jacobi (HJ) reachability, which can compute the \MCnote{backward reachable set (BRS)}, defined as the set of states from which a system is guaranteed to have a control strategy to reach a target set of states. HJ reachability is a powerful tool because BRS can be used for synthesizing both controllers that steer the system away from a set of unsafe states (``safety controllers'') to guarantee safety, and controllers that steer the system into a set of goal states (``goal satisfaction controllers'') to guarantee goal satisfaction. \MCnote{Unlike many formulations of reachability, the HJ formulations are flexible in terms of system dynamics, enabling the analysis of controlled nonlinear systems under disturbances}. Furthermore, HJ reachability analysis \MCnote{is complemented by} many numerical tools readily available to solve the associated HJ partial differential equation (PDE) \cite{LSToolbox, Osher02, Sethian96}. However, the computation is done on a grid, making the problem complexity scale exponentially with the number of states, and therefore with the number of vehicles. Consequently, HJ reachability computations are intractable for large numbers of vehicles. 

In order to accommodate potentially thousands of vehicles simultaneously flying in the air, additional structure is needed to allow for tractable analysis and intuitive monitoring by human beings. An air highway system on which platoons of vehicles travel accomplishes both goals. However, many details of such a concept need to be addressed. Due to the flexibility of placing air highways compared to building ground highways in terms of highway location, even the problem of air highway placement can be a daunting task. To address this, in the first part of this paper, we propose a flexible and computationally efficient method based on \cite{Sethian96} to perform optimal air highway placement given an arbitrary cost map that captures the desirability of having UAVs fly over any geographical location. We demonstrate our method using the San Francisco Bay Area as an example. Once air highways are in place, platoons of UAVs can then fly in fixed formations along the highway to get from origin to destination. \MCnote{The air highway structure greatly simplifies safety analysis, while at the same time allows intuitive human participation in unmanned airspace management.}

A considerable body of work has been done on the platooning of ground vehicles \cite{Kavathekar11}. For example, \cite{McMahon90} investigated the feasibility of vehicle platooning in terms of tracking errors in the presence of disturbances, taking into account complex nonlinear dynamics of each vehicle. \cite{Hedrick92} explored several control techniques for performing various platoon maneuvers such as lane changes, merge procedures, and split procedures. In \cite{Lygeros98}, the authors modeled vehicles in platoons as hybrid systems, synthesized safety controllers, and analyzed throughput. Reachability analysis was used in \cite{Alam11} to analyze a platoon of two trucks in order to minimize drag by minimizing the following distance while maintaining collision avoidance safety guarantees. Finally, \cite{Sabau16} provided a method for guaranteeing string stability and eliminating accordion effects for a heterogeneous platoon of vehicles with linear time-invariant dynamics.

Previous analyses of a large number of vehicles typically do not provide safety and \MCnote{goal satisfaction} guarantees to the extent that HJ reachability does; however, HJ reachability typically cannot be used to tractably analyze a large number of vehicles. In the second part of this paper, we propose organizing UAVs into platoons, which provides a structure that allows pairwise safety guarantees from HJ reachability to better translate to safety guarantees for the whole platoon. With respect to platooning, we first propose a hybrid systems model of UAVs in platoons to establish the modes of operation needed for our platooning concept. Then, we show how reachability-based controllers can be synthesized to enable UAVs to successfully perform mode switching, as well as prevent dangerous configurations such as collisions. Finally, we show several simulations to illustrate the behavior of UAVs in various scenarios.

\MCnote{Overall, this paper is not meant to provide an exhaustive solution to the unmanned airspace management problem. Instead, this paper illustrates that the computation intractability of HJ reachability can be overcome using an air highway structure with UAVs flying in platoons. In addition, the results are intuitive, which can facilitate human participation in managing the airspace. Although many challenges not addressed in this paper still need to be overcome, this paper can provide a starting point for future research in large-scale UASs with safety and goal satisfaction guarantees.}